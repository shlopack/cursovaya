 \section{Расчёт узлов предварительного усиления}

 \textbf{\subsection{Расчет мостового регулятора тембра}}
  \vspace{1em}
 
 Регулятор тембра служит для коррекций частотной характеристики всей схемы, а также приданию звуку желаемой окраски. \par

 Цепочка R1, R2, R3 и C1, C2 – регулятор низких частот. \par
Цепочка R5, С3, С4 – регулятор высоких частот.
Входной каскад усилителя мощности имеет малое входное сопротивление, ограниченное R1. Особенностью пассивного регулятора тембра является то, что эти регуляторы требуют низкого выходного сопротивления предшествующего им каскада и высокого входного сопротивления последующего. По этой причине входной каскад УМ и регулятор тембра РТ отделяют каскадом предварительного усиления, выполненного по схеме с общим коллектором, входное сопротивление которого имеет большую величину, а выходное – малую. Это обеспечивает их совместимость и не потерю сигнала. Нагрузкой для регулятора тембра является входное сопротивление каскада предварительного усиления.

\subsubsection{Коэффициент коррекции:} % (fold)

  \begin{equation}
    \label{eq:equation5_1}
      m = 10^{ \dfrac{\Delta b T_{max}}{20}} = 10^{\dfrac{8}{20}} = 2.5
   \end{equation} 

% subsubsection коэффициент_коррекции_ (end)

\subsubsection{Частота раздела определяется:} % (fold)
  
  \begin{equation}
    \label{eq:equation5_2}
      f_0 = \sqrt{f_{\text{н}} \cdot f_{\text{в}}} = \sqrt{80 \cdot 14000} = 1060~\text{Гц}
   \end{equation} 

% subsubsection subsubsection_name (end)

\subsubsection{Условие неперекрытия зон регулирования:} % (fold)
Одним уз существенных условий нормального функционирования регулятора тембра является расположение частоты нижнего и верхнего среза на расстоянии, обеспечивающим их неперекрытие. Т.е. чтобы избежать взаимного влияние низкочастотного и высокочастотного регуляторов. \par

  \begin{equation}
    \label{eq:equation5_3}
      2 m f_{\text{н}} < f_0 < \dfrac{f_{\text{в}}}{2m} 
   \end{equation} 

% subsubsection subsubsection_name (end)