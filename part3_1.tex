 \section{Расчёт узлов предварительного усиления}

 \textbf{\subsection{Расчет мостового регулятора тембра}}
  \vspace{1em}
 
 Регулятор тембра служит для коррекций частотной характеристики всей схемы, а также приданию звуку желаемой окраски. \par

 Цепочка R1, R2, R3 и C1, C2 – регулятор низких частот. \par
Цепочка R5, С3, С4 – регулятор высоких частот.
Входной каскад усилителя мощности имеет малое входное сопротивление, ограниченное R1. Особенностью пассивного регулятора тембра является то, что эти регуляторы требуют низкого выходного сопротивления предшествующего им каскада и высокого входного сопротивления последующего. По этой причине входной каскад УМ и регулятор тембра РТ отделяют каскадом предварительного усиления, выполненного по схеме с общим коллектором, входное сопротивление которого имеет большую величину, а выходное – малую. Это обеспечивает их совместимость и не потерю сигнала. Нагрузкой для регулятора тембра является входное сопротивление каскада предварительного усиления.

\subsubsection{Коэффициент коррекции:} % (fold)

 \begin{equation}
  \label{eq:equation5_1}
  m \geq 10^{ \left| \Delta b_{\text{т}}  
   \right| /20 }
\end{equation}
\begin{equation*}
  m \geq 10^{ \left| \pm 14.0  \right| /20 } = 5.012
\end{equation*}

% subsubsection коэффициент_коррекции_ (end)

\subsubsection{Частота раздела определяется:} % (fold)
  
  \begin{equation}
    \label{eq:equation5_2}
      f_0 = \sqrt{f_{\text{н}} \cdot f_{\text{в}}} = \sqrt{80 \cdot 14000} = 1060~\text{Гц}
   \end{equation} 

% subsubsection subsubsection_name (end)

\subsubsection{Условие неперекрытия зон регулирования:} % (fold)
Одним уз существенных условий нормального функционирования регулятора тембра является расположение частоты нижнего и верхнего среза на расстоянии, обеспечивающим их неперекрытие. Т.е. чтобы избежать взаимного влияние низкочастотного и высокочастотного регуляторов. \par

  \begin{equation}
    \label{eq:equation5_3}
      2 m f_{\text{н}} \leq f_0 \leq \dfrac{f_{\text{в}}}{2m} 
 \end{equation} 
 \begin{equation}
   \label{eq:equation5_4}
     400 \leq f_0 \leq 2800
  \end{equation} 

    Т.о. видно, что не происходит взаимного влияния низкочастотного и высокочастотного регулятора.

% subsubsection subsubsection_name (end)

\subsubsection{Сопротивления подстроечных резисторов R = R2 = R5:}
\begin{equation}
   \label{eq:equation5_5}
R=0.5 \cdot R_{\text{ВХ СЛ}}=0.5 \cdot R_{\text{ВХ КПУ 2}}=0.5 \cdot cdot 10^6=100~\text{кОм}
\end{equation} 

\subsubsection{ Номиналы регистров регулирования НЧ:}
\begin{equation}
   \label{eq:equation5_6}
   R_1=\dfrac{R}{m}=100 \cdot 10^3 /4 =25~\text{кОм}
   \end{equation} 
   \begin{equation}
   \label{eq:equation5_7}
   R_3=\dfrac {R_1}{m}=25 \cdot 10^3/4=6250~\text{Ом}
   \end{equation} 
   \subsubsection{ 6. Сопротивление буферного  резистора R4.}
   Буферный резистор обеспечивает развязку  низкочастотного и высокочастотного регуляторов. 
   \begin{equation}
   \label{eq:equation5_6}
   R_4=(0.05 \ldots 0.1) \cdot R=(0.05 \ldot 0.1) \cdot 100 \cdot 10^3 = 5000~\text{Ом}
   \end{equation} 
   \subsubsection{    Задание номинальных значений емкостей.}
   Емкости в схеме регулятора тембра обеспечивают его работу как фильтра.
   \begin{equation}
   \label{eq:equation5_7}
   C_1=\dfrac{1}{2 \cdot \pi \cdot 60 \cdot 4 \cdot 100 \cdot 10^3}=66~\text{нФ}
   \end{equation} 
   \begin{equation}
   \label{eq:equation5_8}
   С_2=m \cdot C_1=5 \cdot 5 \cdot 0.27= 264~\text {нФ}
   \end{equation} 
        Емкости С3 и С4 формируют ВЧ-регулятор.
        \begin{equation}
   \label{eq:equation5_9}
   C_3=\dfrac {m^2}{(4\cdot \pi \cdot f_{\text{в}}\cdot m \cdot R) }= 4^2/(4 \cdot \pi \cdot 12000 \cdot 100 \cdot 10^3)=1~\text{нФ}
   \end{equation} 
  \begin{equation}
   \label{eq:equation5_10}
 C_4=m \cdot C_3=4 \cdot 1 \cdot 10^-9=4~\text{нФ}
\end{equation} 
\subsubsection{   Задание входного и выходного сопротивления регулятора тембра.   }
\begin{equation}
   \label{eq:equation5_11}
R_{\text{ВХ T}}=R_1+R_3=25000+6250=31250~\text{Ом}
\end{equation} 
\begin{equation}
   \label{eq:equation5_12}
   R_{\text{ВХ Т}}=R_4+\defrac {R_1 \cdot R_3}{R_1+R_3} = 5000+25000 \cdot 6250 /(2500+6250)=10~\text{кОм}
   \end{equation}

\subsubsection{ Определение требования к выходному сопротивления предыдущего каскада (КПУ 1)   }

\begin{equation}
   \label{eq:equation5_12}
   R_{\text{ВЫХ ПРЕД}}=0.2 \cdot R_{\text{ВХ Т}}=0.2 \ cdot 31250=6250~\text{Ом}
\end{equation}
Выходное сопротивление предшествующего каскада для регулятора тембра представляет собой эквивалентное сопротивление генератора сигнала. Для согласования каскада и передачи сигнала с минимальными потерями оно выбирается в 5…10 раз меньше входного сопротивления регулятора тембра.

\subsubsection{ Определение положения движков R2 и R5, соответствующие линейной частотной характеристике:}

  \begin{equation}
  \label{eq:equation5_12}
  R^,,=\dfrac{R \cdot m}{m^2-1}=100 \cdot 10^3 \cdot  4/(16-1)=26670~\text{Ом}
  \end{equation}

  \begin{equation}
  \label{eq:equation5_13}
  R^,=R-R^,,=100 \cdot 10^3 -26670=73~\text{кОм}
  \end{equation}

  \subsubsection{ Номинальный коэффициент передачи тембра на средних частотах имеет вид:}
   
  \begin{equation}
  \label{eq:equation5_14}
   K=\dfrac {R_3}{R_1+R_3}=6250/(2500+6250)=0.2
  \end{equation}

  Регулировка тембра на НЧ осуществляется за счет резисторов R3 и R1. Они определяют основное изменение напряжения генератора до уровня непосредственно усилителя мощности. Предел регулировки тембра и характеризует изменение АЧХ тембра, т.е. усиление по напряжению.

  \subsubsection{  Номинальное входное напряжение РТ:}

  \begin{equation}
  \label{eq:equation5_12}
   U_{\text{ВХ Т}}=\dfrac{U_{\text{ВХ СЛЕД}}}{K}=0.016/0.2=0.08~\text{В}
  \end{equation}


   