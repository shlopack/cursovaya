 \section{Синтез устройств фазовой синхронизации}

 \textbf{\subsection{Алгоритм автоматизированного проектирования УФС}}
  \vspace{1em}
  Схемы автоматизированного проектирования УФС предствалены на рис 3.1. На первом этапе проектирования операясь на условия технического задания, следует выбрать структурную схему устройства(блок 1) определить многомерную область устойчивости(блок 2) и определить диапазон изменения искомых параметров(блок 3) проектируемого устройства. Этат этап основывается на анализе линеаризованной модели системы, где происходит определение граничных значений параметров системы из условия обеспечения устойчивости "в малом". По результатом анализа получается n-мерная область параметров, внутри которой осуществляется поиск. 
  На втором этапе производится построение областей качества по времени переходных процессов, т.е. оценивается динамика системы при заданных расстройках частоты или коэффициентов деления цепи обратной связи. Для моделирования переходных процессов используются математические модели, которые представлены автором в главе 2. Также производится построение областей качества по дисперсии шума. Следует отметить, что области качества по скорости переходных процессов представляют собой многомерные матрицы, элементы которых выражают время переходных процессов в условных еденицах в зависимости от параметров проектируемой системы, а области качества по дисперсии шума - это многомерные матрицы, элементы которых выражают дисперсию выходного сигнала УФС в относительных еденицах, при воздействии малых шумов (шумов, дисперсия которых значительно меньше квадрата величин полезного сигнала) в зависимости от параметров проектируемой системы. Построение областей качества во времени переходных процессов осуществляеся блоком 6, а построение областей качества по дисперсии шума - блоком 5.
  Процесс синтеза УФС определяется функциональным назначением разрабатываемого устройства [,]. В техническом задании на разработку устройства предъявляются требования к харрактеристикам, которые можно условно разделять на требования к динамике системы быстродействию, полосе захвата и др.; и требования к статистическим харрактеристикам - дисперсии шумов выходного сигнала, вероятность удержания синхронного режима, среднее время срыва синхронизма и др. Исходя из технических харрактеристик технического задания можно выделить четыре вида проектируемых устройств. [].
  \textbf{\subsection{Первый вид УФС}}
  Усторойсва у которых существуют требования к техническим характеристикам по быстродействию и качеству переходных процессов, по ширине полосы захвата и синхронизма, но отсутствуют ограничения на статистические характериститки устройства. К таким устройствам относятся стабилиаторы напряжения и тока, регуляторы мощности, системы регулировани скорости вращения вала электродвигателя и др.
    \textbf{\subsection{Второй вид УФС}}
  Устройства, у которых существуют требования к ЦЕНТРАЛЬНОЙ частоте выходного сигнала и статистическим характеристтикам, но отсутствуют требования к динамическим свойствам устройства. К таким устройствам относятся синтезаторы частот, опорные генераторы частот, опредеенный класс радиоприемных устройств.
    \textbf{\subsection{Третий вид УФС}}
    устройства, у которых существуют требования к характеристикам по быстродействию, качеству переходных процессов, по ширирне полосы захвата и синхронизма, а также сформулированы требования к спектральным хараактеристикам выходного сигнала т качеству работы при воздействии шумов. При этом динамические характеристики являются приоритетными , а статистические - дополнительными. К УФС третьего вида относятся : прецизионные стабилизаторы напряжения и тока, синхронизаторы , умножители частот, следящие системы радиоприемных и передающих устройств, системы слежения за скоростью и положением движущихся оббъектов и др.
    \textbf{\subsection{Четвертый вид УФС}} 
    Устройства у которых описаны требования к спектральным характеристикам выходного сигнала, статистическим характеристикам и эти требования являются для разрабатываемого устройства приоритетными. Также существуют требовани к  динамическим характерисикам, но они рассматриваются как дополнительные. К УФС четвертого вида относятся - детекторы частоты и фазы, синтезаторы частот, следящие фильтры, синхронизаторы информационных сетей, системы слежения за частотой и фазой ыходного сигнала, регистраторы сигналов и др.
    Процесс проектирования на втором этапе осуществляется для четырех видов УФС различным образом.
    Для УФС первого вида проектирования организовано блоками: 4-6-8-11-12-13-14-16-17.
    Для УФС второго вида при проектировании задействованы блоки № : 4-5-7-9-11-12-13-15-18-19-20.
    Для УФС третьего вида при проектировании используются следующие блоки 4-6-8-10-11-12-13-15-18-19-20.
    Для УфС четвертого вида при проектировании задействованы блоки : 4-5-7-9-11-12-13-15-18-19-20; или возможен вариант для следующих систем: 4-5-7-9-11-12-13-14-16-17.
    На третьем этапе проектирования происходит анализ рекомендуемых параметров  проектируемой системы, выбор элементов сструктурной схемы, проверка пунктов технического задания и, в слудчае соответствия требования ТЗ, документирование выбранных элементов и характеристик устройства. Третий этап выполняется блоками 21-22-23-24. В блоке 21 требования к постонным времени, коэффицентом усиления анализируются и производится выбор соответствующих микросхем, резисторов, емкостей и др. Блок 22 осуществляет моделирование статичесих, динамических и статистических характеристи по модели электрической схемы на основе стандартныхпакетов прикладных программ( SPICE,     ,     ).
    Если характеристики не отвечают требуемым то рекомендуется изменить структуру или вид модуляции и перейти к блоку один 1, после чего повторить процесс проектирования. При выполнении всех требовании ТЗ процесс проектирования завершается выводом информации о выбранных элементах спроектировнного устройства.