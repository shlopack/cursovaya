\textbf{\subsection{Расчет входного каскада}}
\vspace{1em}

Входной каскад выполнен на дифференциальном каскаде. Дифференциальный каскад характеризуется тем, что усиление по напряжению при симметричном съеме сигнала равен коэффициенту усиления в схеме с общим эмиттером. Как и предоконечный каскад, входной является усилителем по напряжению. Однако из-за ограниченного сопротивления в коллекторной цепи коэффициент усиления по напряжению дифференциального сигнала не будет достигать больших значений. При этом синфазный сигнал подавляется значительно, что является уменьшением синфазных помех. 
За счет отсутствия местной обратной связи в дифференциальном каскаде достигается большое увеличение по напряжению.  Что непосредственно влияет на петлю ООС в усилителе мощности, увеличивая коэффициент усиления в петле ООС.
Транзисторы VT1 и VT2 работают в режиме А, который обеспечивает усиление по напряжению, но не дает большого КПД.

VT1 и VT2 – дифференциальный каскад.
R1 – сопротивление базового делителя. Ограничивает входное сопротивление каскада.
R3 – сопротивление эмиттерной цепи.
R2 – сопротивление коллекторной цепи. Задает нагрузку каскада.
R5, R4 и C1 – цепь обратной связи. По постоянному току 100\%, по переменному определяется сопротивлением резистора R4.

\subsubsection{Ток покоя коллектора VT1 и VT2:}
\begin{equation}
\label{eq:equation4_1}
 I_{\text{0 К1}} = (5 \ldots 10) I_{\text{Б m3}} / h_{\text{21 3}} = 7 \cdot 0.024 / 80 = 2.1~\text{мА} 
\end{equation}

\subsubsection{Параметры выбора транзисторов:}