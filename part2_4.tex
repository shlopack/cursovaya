\textbf{\subsection{Универсальная математическая модель устройств фазовой синхронизации}}
\vspace{1em}
Основываясь на анализе аналоговых, импульсных и цифровых систем фазовой синхронизации, автором делается вывод о возможности создания универсальной математической модели УФС. Для этого в схемах устройств (рис. 2.1,2.3,2.5,2.6) выделяются одинаковые с точки зрения математического описания блоки: НЛЧ, состоящая из звеньев фильтрации и коррекции и ОУ; цепи обратной связи. Основным изменяющимся блоком является фазовый детектор. Именно этот блок определяет, каким будет устройство: аналоговым, импульсным либо цифровым; также он определяет вид модуляции, играет большую роль при выборе шага моделирования и при расчёте переходных процессов.\par
Таким образом, неизменными остаются уравнения состояния~(\ref{eq:equation2_5}), описывающие звенья фильтрации и коррекции; система уравнений разомкнутого устройства фазовой синхронизации:\par
\begin{equation}\label{eq:equation2_24}
  X(t)=\phi(t-t_n)X(t_n)+\int^t_n\phi(t-\lambda)B\epsilon(t)d\lambda
\end{equation}\par
Далее необходимо выбрать вид фазового детектора, который определит надлежащее математическое описание:
\begin{itemize}
\item для АУФС, в зависимости от выбранного детектора, задаётся соответствующая аналитическая зависимость (например,~(\ref{eq:equation2_1})) [];
\item  для ИУФС в случае совместной амплитудно-широтно-частотно-импульсной модуляции (АИМ-ШИМ-ЧИМ) уравнения сравнивающего устройства имеют вид:\par
    \begin{equation}
    \label{eq:equation2_25}
    \epsilon(t)=
    \begin{cases}
    h_n,t\in [t_n;t_n +\tau_n],
    \\-h_n,t\in [t_n+\tau_n ;t_{n+1}]
    \end{cases}
    \end{equation}
    \end{itemize}
    Здесь описана двухполярная модуляция второго рода[]; $h_n$- амплитуда воздействующего импульса на $n$-ом такте квантования; $t_n$- момент появления импульса положительной полярности (начало отсчёта на заданном шаге); $\tau_n=t_{n+1}+t_n$ – длительность периода на n-ом шаге.
- для ЦУФС при заданном виде дискриминационной характеристики ФД (например, рисунок~2.7в) алгоритм моделирования следующий:
1. Определяются разрешённые значения цифровых отсчётов сигнала рассогласования:\par
$\delta\phi=\frac{2\pi}{L}$,где $\delta\varphi_n$ – фазовые рассогласования на $n$-ом шаге моделирования; $L$ – разрядность АЦП.\par
2. Определяется амплитуда сигнала рассогласования:
\begin{equation}
\label{eq:equation2_26}
  h=h_{max}\sin(\varphi_0+\varphi_n),
\end{equation}\par
\noindent где $h$ – текущее неразрешённое значение сигнала рассогласования; $h_{max}$ – максимальное значение сигнала рассогласования; $\varphi_0$ – значение начальной фазы выходного сигнала на предыдущем шаге; $\varphi_n$ – набег фазы на n шаге расчёта.\par
3. Нахождение разрешённого значения сигнала рассогласования $h_n$:
\begin{equation}
\label{eq:equation2_27}
  h_i\geq h\geq h_{i+1} \Rightarrow h_n = h_i,
\end{equation}\par
\noindent где $h_i, h_{i+1}$ – разрешённые значения сигнала рассогласования, определяющиеся разрядностью преобразователя АЦП и видом дискриминационной характеристики (рисунок 2.7).\par
Следующим шагом после выбора фазового детектора является описание уравнения замыкания, которое в общем виде выглядит следующим образом:\par
\begin{equation}
\label{eq:equation2_28}
  \int^{t_{nk}}_{t_0}\omega(t)dt = 2\pi N_{oc}
\end{equation}\par
Уравнение~(\ref{eq:equation2_28}) для универсальной модели решается н интервале времени [$t_0;t_{nk}$], при этом весь интервал разбивается на различное количество шагов решения:\par
\begin{itemize}
\item для АУФС количество шагов решения определяется от требуемой точности (3-6 шагов – погрешность до $10$, до 100 шагов – погрешность не более 1);
\item для ИУФС количество шагов составляет 1-2 шага, при этом погрешность не более 3;
\item для ЦУФС количество шагов различно и определяется частотой стробирования ФД $f_c$ (рисунок 2.8), что составляет не менее 20 шагов и погрешность не более 2.
\end{itemize}\par
Для всех устройств определяется набег фазы выходного сигнала:
\begin{equation}
\label{eq:equation2_29}
 \varphi_n=\frac{1}{N_{oc}}\int^{nT+\delta t_n}_{nT}\omega(t)dt,
\end{equation}\par
\noindent где $\delta t_n$ – интервал времени расчёта $ \varphi_n$- определяется из величины шага решения. \par
В результате получена модель, которая описывает УФС с любым видом модуляции. Предложенная модель может достраиваться на основе математического описания фазовых детекторов иного типа с различным видом фазовых характеристик и дополнительными функциональными свойствами. \par
Описание новых блоков ФД, ОУ может производиться аналитически, задаваться алгоритмическими схемами или таблицами.\par
К недостаткам такой модели можно отнести тот факт, что при требовании высокой точности расчёта, больших постоянных времени звеньев фильтрации и высокой частоте стробирования модель использует большие временные вычислительные ресурсы. Для устранения этого недостатка автор предлагает использовать аналитическое решение для типовых звеньев фильтрации, поскольку расчёт полученных формул позволяет наряду с высокой точностью затрачивать небольшие расчётные ресурсы. Неудобством данного подхода является необходимость создания библиотеки для наиболее распространённых звеньев[].\par
Сочетание имитационного моделирования и моделирования на основе аналитического описания позволило создать универсальную модель устройств фазовой синхронизации, которая наряду с требуемой (заранее задаваемой) точностью расчёта отличается высокой эффективностью, быстродействием и позволяет в случае необходимости достраивать модель в соответствии с требованиями разработчика.\par
Выводы:
\begin{itemize}
\item  Проведён обзор методов построения математических моделей устройств фазовой синхронизации. Разработаны обобщённые функциональные схемы для аналоговых, импульсных и цифровых УФС.
 \item Предложен способ математического описания аналоговых, импульсных и цифровых УФС, отличающийся от существующих подходом к математическому описанию основных блоков системы и позволяющий с требуемой точностью (заранее задаваемой) производить расчёт переходных процессов.
 \item Впервые разработана универсальная математическая модель устройств фазовой синхронизации, позволяющая моделировать процессы в УФС с учётом вида модуляции нелинейных свойств в синхронном и асинхронном режимах.
Основные научные результаты настоящей главы приведены в следующей литературе:
\end{itemize}
