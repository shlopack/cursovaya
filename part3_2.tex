\textbf{\subsection{Расчет  каскада предварительного усиления}}
  \vspace{1em}
  
  Каскад предварительного усиления обеспечивает усиление входного сигнала до необходимого уровня для действия усилителя мощности. Также он выполняет функцию промежуточного звена между регулятором усиления и регулятором тембра.\par
  В нашем случае Uн и R н являются входными параметрами регулятора тембра. При двуполярном питании оконечного каскада в качестве источника Ео может использоваться любая из половинок.

  \subsubsection{Амплитуда напряжения и тока нагрузки:}

  \begin{equation}
  \label{eq:equation6_1}
    U_{\text{нм}} = U_{\text{н}} \cdot \sqrt{2} = 0.019 \cdot \sqrt{2} = 0.027~\text{В};
  \end{equation}

  \begin{equation}
  \label{eq:equation6_2}
    I_{\text{нм}} = U_{\text{нм}} / R_{\text{н}} = 0.027 / 2854 = 9.5~\text{мкА}.
  \end{equation}  

  \subsubsection{Ток покоя:}  

  При дальнейшем расчете параметров выбора транзистора необходимо учесть тот факт, что нагрузка подключена в коллекторную цепь. Но по той причине, что ток коллектора будет равен току эмиттера, пренебрегая базовым, в обозначении сразу указан ток коллектора.\par

  \begin{equation}
  \label{eq:equation6_3}
    I_{\text{ок}} \geq (5 \ldots 10) \cdot I_{\text{нм}} = 10 \cdot 9.5 \cdot 10^{-6} = 95~\text{мкА}.
  \end{equation} 

  Который не удовлетворяет условию тока, необходимому для раскачки УМ: $I_{\text{ок}} = 0.5 \ldots 2$мА. Выбираем ток покоя $I_{\text{ок}} = 0.5$ мА.

  \subsubsection{Напряжение коллектор-эмиттер транзистора:}  

  \begin{equation}
  \label{eq:equation6_4}
    U_{\text{кэ}} \geq U_{\text{Нm}} + U_{\text{КЭmin}} = 0.027+2 = 2.027~\text{В}.
  \end{equation}   

  При этом $U_{\text{КЭmin}} = 1 \ldots 2$ В.
  Амлитуда сигнала мала, поэтому выбираем $U_{\text{КЭ}} = 4 $В.

  \subsubsection{Напряжение источника питания:}  

  \begin{equation}
  \label{eq:equation6_5}
    E_{\text{0П}} \geq (2 \ldots 3) U_{\text{КЭ}} = 3 \cdot 4 = 12~\text{В};
  \end{equation} 

  \begin{equation}
  \label{eq:equation6_6}
    E_{\text{0}} = (1.2 \ldots 1.3) E_{\text{0П}} = 1.3 \cdot 12 = 14.4~\text{В}.
  \end{equation} 

  \subsubsection{Сопротивление в цепи эмиттера:}  
  
  \begin{equation}
  \label{eq:equation6_7}
    R_{\text{Э}} = R_4 + R_5 = \dfrac{U_{\text{Э}}}{I_{\text{0Э}}} = 0.3 \cdot \dfrac{E_{\text{0П}}}{I_{\text{0К}}} = 7.2~\text{кОм}.
  \end{equation}

  \subsubsection{Определяем сопротивление R3:} % (fold)

  \begin{equation}
  \label{eq:equation6_8}
    R_3 = \dfrac{ E_{\text{0П}} - U_{\text{КЭ}} - U_{\text{Э}}}{I_{\text{0К}}} = \dfrac{12- 44 -3.6}{0.5\cdot 10^{-3}} = 8.8~\text{кОм}.
  \end{equation}

  % subsubsection определяем_сопротивление_r3_ (end)
  \subsubsection{Амплитуда тока:} % (fold)
  
  \begin{equation}
  \label{eq:equation6_9}
    I_{\text{Кm}} = \dfrac{U_{\text{нм}}}{R_3 \cdot R_{\text{н}}} = \dfrac{0.027 \cdot 11.7}{8.8 \cdot 2854} = 0.013~\text{мА}.
  \end{equation}
  % subsubsection _амплитуда_тока_ (end)
  
  \subsubsection{Мощность, рассеиваемая на коллекторном переходе:} % (fold)

  \begin{equation}
  \label{eq:equation6_10}
    P_{\text{К}} = U_{\text{кэ}} cdot I_{\text{0К}} = 4 \cdot 0.5 \cdot 10^{-3} = 2~\text{мВт}. 
  \end{equation}
  
  % subsubsection мощность_рассеиваемая_на_коллекторном_переходе_ (end)

  \subsubsection{Критерии выбора транзистора:} % (fold)

  \begin{equation}
  \label{eq:equation6_11}
    P_{\text{К доп}} \geq (1.1 \ldots 1.3) P_{\text{К}} = 1.3 \cdot 2 \cdot 10^{-3} = 2.6~\text{мВт};
  \end{equation} 

  \begin{equation}
   \label{eq:equation6_12}
     U_{\text{КЭ}} \geq E_0 = 12~\text{В};
  \end{equation} 

  \begin{equation}
  \label{eq:equation6_13}
    I_{\text{К доп}} \geq (1.1 \ldots 1.3)(I_{\text{0К}} + I_{\text{Кm}}) = 1.3 \cdot (0.5 \cdot 10^{-3} + 0.013 \cdot 10^{-3}) = 0.66~\text{мА}; 
  \end{equation} 

  \begin{equation}
   \label{eq:equation6_14}
     f_{h_{21}} \geq (20 \ldots 30) f_{\text{в}} = 30 \cdot 14000 = 420~\text{кГц}
  \end{equation} 
  % subsubsection критерии_выбора_транзистора_ (end)