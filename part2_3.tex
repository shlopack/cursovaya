\textbf{\subsection{Цифровые устройства фазовой синхронизации}}
\vspace{1em}
В соответствии с принятой классификацией [],[],[] цифровые устройства фазовой синхронизации (ЦУФС) можно различать по месту расположения в их структуре аналого-цифрового преобразователя (АЦП). Если АЦП1 расположен на входе устройства до фазового детектора (рисунок 2.5), то цифровой фазовый детектор (ЦФД) представляет собой устройство сравнения двух цифровых последовательностей, результатом которого является цифровой сигнал $\epsilon(t)$, несущий информацию о фазовом рассогласовании входного сигнала и сигнала обратной связи []. Устройство управления (УУ) осуществляет формирование сигнала воздействия на объект управления, при этом необходимо преобразования цифрового сигнала в аналоговый $e(t)$, реализуемое цифро-аналоговым преобразователем (ЦАП). Выходной сигнал $\omega(t)$ после прохождения цепи обратной связи преобразуется в цифровой сигнал с помощью АЦП2. Основными недостатками данного варианта являются следующие: наличие запрещённых частот на входе (частот вне диапазона преобразования АЦП1); увеличение ошибки устройства в целом, т.к. АЦП1 не охвачен обратной связью системы.\par
При расположении АЦП внутри петли устройства (рисунок 2.6) на входы фазового детектора поступают аналоговые или импульсные сигналы $y(t)$ и $u(t)$. Сигнал фазового рассогласования преобразуется к цифровому виду с помощью АЦП, расположенного внутри петли. Такая архитектура ЦУФС не имеет вышеуказанных недостатков, и поэтому получила большее распространение [],[].\par
// РИСУНОК 2.5
// РИСУНОК 2.6\par
При математическом описании ЦУФС следует обратить внимание на вид дискриминационной характеристики фазового детектора [], приведённой на рисунке 2.7. Из рисунка видно, что форма характеристик может быть различной и формируется при проектировании ЦУФС. Наилучшей дискриминационной характеристикой с точки зрения точности и быстродействия для обоих вариантов является ступенчато-синусоидальная [],[]. Существует большое количество схемотехнических решений и программных алгоритмов для создания ЦФД [],[],[]. Программные средства организации ЦФД обладают значительным преимуществом в плане реализации вида дискриминационных характеристик, но на современном этапе развития цифровой техники не получили широкого распространения из-за малого быстродействия [].\par
// РИСУНОК 2.7\par
В качестве примера рассмотрим один из вариантов организации ЦУФС. Выберем ЦФД [] с внешним сигналом синхронизации (рисунок 2.8а), где Ф1 и Ф2 – формирователи импульсов;$ T~-~RS$ – триггер;$ f_c $– частота стробирования, которая диктуется требованиями к точности системы. На рисунке 2.8б показаны временные диаграммы работы ЦФД.\par
// РИСУНОК 2.8\par
Сигналы $e^*(t)$ и $e(t)$ – управляющие сигналы, которые могут быть использованы в том или ином виде[]. На рисунке 2.6 блок управления (УУ) позволяет производить преобразования сигнала на цифровом уровне, включая микроконтроллерные устройства[],[].
Таким образом, если частота стробирования в АЦП-ЦАП выше в 10 и более раз, чем максимальная рабочая частота сигналов $y(t)$ и $u(t)$, то при описании фазового детектора, УУ и ЦАП нет необходимости усложнять математическую модель и вводить дополнительное описание работы этих блоков.\par
Такой подход вполне оправдан [],[],[] и соответствует рабочим режимам ЦУФС, когда входная частота меньше в 64, 128 или даже 256 раз. Но стремление увеличить рабочую частоту сигналов $y(t)$ и $u(t)$ (с целью увеличения быстродействия, расширения диапазона частот, применения решающих микроконтроллерных устройств в петле регулирования и др.) приводит к тому, что при работе на граничных частотах это соотношение существенно уменьшается, как и точность математических моделей в этих режимах. Автором предлагается составлять математическую модель с минимальным отрезком интегрирования, равном периоду сигнала стробирования[]. Это позволит сделать модель точной в любых режимах работы; платой за точность будет являться увеличение времени анализа.\par
Для вывода разностных уравнений, описывающих поведение ЦУФС, определим выходную координату $z(t)$. Будем считать на рисунках 2.5 и 2.6, что часть прямой цепи – звенья фильтрации и коррекции, объект управления – непрерывная линейная часть. Прохождение сигнала через НЛЧ описывается уравнениями~(\ref{eq:equation2_5}-\ref{eq:equation2_8}). Для определённости примем характеристику цифрового фазового детектора линейной (рисунок 2.7а), тогда:\par
\begin{equation}
\label{eq:equation2_19}
\epsilon=h_n=\Delta\cdot int\left[ \frac{\varphi_n}{\Delta\varphi_n}\right]\grave{}
\end{equation}
\noindent где $\Delta\varphi_n =  \frac{2\pi}{l}$, $L$ – разрядность преобразования АЦП,$ \varphi_n$- набег фазы выходного сигнала за время квантования $\delta t_n$(рисунок 2.8). \par
Уравнение разомкнутого контура на интервале времени $\delta t_n$: \par
\begin{equation}
\label{eq:equation2_20}
X(t)=\Phi(t-nT)(x(nT)+A^{-1}Bh_n)-A^{-1}Bh_n.
\end{equation}\par
На каждом шаге квантования производится расчёт набега фазы выходного сигнала:\par
\begin{equation}
\label{eq:equation2_21}
\varphi_n=\frac{1}{N_{oc}}\int_{nT}^{nT+\delta t_n}\omega(t)dt,
\end{equation}
\noindent где $\omega(t)=z(t)+g(t)=M(e(t))+g(t); M(e(t))$-модуляционная характеристика объекта управления. \par
Далее производится определение значения начальной фазы выходного сигнала $\omega(t)$, для чего от значения времени $t_0$, при котором $\varphi_n=0$, происходит суммирование значений набега фазы сигнала за расчётное количество тактов квантования. При этом проверяется неравенство:
\begin{equation}
\label{eq:equation2_22}
\sum^{n}_{\theta=1}\varphi_n<2\pi N_{oc}
\end{equation}
\noindent  где $\theta$ – первый шаг квантования, при котором $\varphi_n=0$; $n$ – текущий шаг квантования. При выполнении условия \par
\begin{equation}
\label{eq:equation2_23}
\sum^{n}_{\theta=1}\varphi_n<2\pi N_{oc}
\end{equation}\par
\noindent делается вывод о величине фазового рассогласования входного и выходного сигналов, после чего определяется значение $h_n$ на последующий такт. При этом в зависимости от организации ЦФД возможны различные алгоритмы воздействия сигнала $h_n$ по времени (рисунок 2.8 $e^*,e$). \par
Установившийся режим в ЦУФС характеризуется неизменностью $\varphi_n, z_n$.\par
Таким образом, автором предлагается математическая модель описания ЦУФС, которая имеет общие принципы описания для всех видов устройств фазовой синхронизации.
