\textbf{\subsection{Расчет предоконечного каскада}}
\vspace{1em}

Предоконечный каскад, как и входной, являются усилителями напряжения до уровня, необходимого в непосредственном усилителе мощности, выполненном на оконечном каскаде, а также для согласования входного сигнала с входом оконечного каскада. Поэтому в предоконечном каскаде не ставится задача усиления мощности и он работает в режиме А, с КПД порядка 25\%. В этом режиме ток в выходной цепи протекает в период всего действия сигнала. Режим А дает возможность получения максимальной амплитуды выходного сигнала с минимальными искажениями. Воздействие на низкоомную нагрузку RН сигнала большой амплитуды приводит к значительному увеличению КПД усилителя и его мощности.
За счет включения в коллекторную цепь предоконечного каскада, выполненного по схеме с общим эмиттером, динамической нагрузки получаем большой коэффициент усиления по напряжению. \par
Каскад охвачен местной положительной ОС, что дает возможность увеличения коэффициента усиления K, но уменьшения полосы пропускания.\par
Таким образом, основные особенности каскада предварительного усиления в том, что за счет работы в режиме А, он обеспечивает минимальные искажения, при достаточно усилении амплитуды сигнала. Включение в выходную цепь динамического сопротивления позволяет увеличить коэффициент усиления K в десятки раз.\par


\subsubsection{Ток покоя транзистора VT3:}

\begin{equation}
\label{eq:equation3_1}
 I_{\text{O К3}} = (2\ldots 3)\cdot I_{\text{Б m4}}  = (2\ldots 3)\cdot I_{\text{нм}}/h_{\text{21 экв}}
\end{equation}
\begin{equation*}
 I_{\text{O К3}} = \dfrac{2.5 \cdot 53.25 }{ 254 } = 55.55
\end{equation*}
 
 \subsubsection{Выбор резистора R7:} 
 
\begin{equation}
\label{eq:equation3_2}
 R_{7} = (30\ldots 50)\cdot R_{н} = 40 \cdot 4 = 100~\text{Ом}
\end{equation}
Резистор R7 включается в цепь для того, чтобы не закорачивать источник питания конденсатором С3, обеспечивающим включение в выходную цепь транзистора динамической нагрузки R6. Основное усиление напряжения происходит за счет динамической нагрузки R6, потому резистор R7 выбирается малой величины.

\subsubsection{Выбор резистора R6:} 
\begin{equation}
\label{eq:equation3_3}
 R_{6} = (E_{0}– U_{\text{БЭ5}} – I_{\text{О К3}} \cdot R_7)/I_{\text{0 К3}} = (15 – 0.6 – 0.023 \cdot 100)/0.02 = 100~\text{Ом}
\end{equation}

При прохождении сигнала динамическое сопротивление R6 будет определяться:
\begin{equation}
\label{eq:equation3_4}
 R_{\text{6Д}} = \dfrac{R_6}{1-K_{\text{ОК}}} = 10 \cdot 542.212 = 100~\text{Ом}
\end{equation}

Коэффициент усиления оконечного каскада $K_{\text{ОК}}$, т.к. он является повторителем напряжения, близок к 1 и составляет более 0.9: $K_{\text{ОК}}$ = 0.9. \par
С обеих сторон резистора R6 потенциалы близки за счет того, что цепь термостабилизации не вносит особо падения напряжения и транзисторы VT4 и VT5 являются повторителями напряжения. Ввиду этого на обоих концах установятся близкие потенциалы, т.е. разность потенциалов будет очень мала и ток практически не будет протекать. Что эквивалентно включению большого сопротивления. За счет этого происходит увеличения коэффициента усиления. \par

\subsubsection{Определение емкости C3:} 
Эта емкость устраняет протекание переменного тока по цепи R6 – R7 – земля и увеличивает коэффициент усиления каскада. Обеспечивает связь транзистора VT3 с нагрузкой R6 через оконечный каскад. \par
\begin{equation}
\label{eq:equation3_5}
 C_3 \geq \dfrac{5 \ldots 10}{2 \pi f_{\text{н}} (R_7 + R_{\text{н}})} geq \dfrac{5}{2 \pi 18 (52 + 52)} geq 654~\text{мкФ}
\end{equation}

\subsubsection{Параметры выбора транзистора VT3:}
\begin{equation}
\label{eq:equation3_6}
P_{\text{к доп}} = (1.2 \ldots 1.5) P_{\text{к 3}} = (1.2 \ldots 1.5) \cdot \dfrac{E_0 I_{\text{0 кз}}}{2} = 1.3 \cdot \dfrac{16 0.02}{2} = 0.025~\text{Вт}
\end{equation}

\begin{equation}
\label{eq:equation3_7}
I_{\text{к 3m}} = I_{\text{О КЗ}} + \dfrac{I_{\text{нм}}}{h_21} = 0.013 + \dfrac{2.54}{600} = 0.024~\text{А}
\end{equation}
