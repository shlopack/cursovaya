\textbf{\subsection{Расчет предоконечного каскада}}
\vspace{1em}

Предоконечный каскад, как и входной, являются усилителями напряжения до уровня, необходимого в непосредственном усилителе мощности, выполненном на оконечном каскаде, а также для согласования входного сигнала с входом оконечного каскада. Поэтому в предоконечном каскаде не ставится задача усиления мощности и он работает в режиме А, с КПД порядка 25\%. В этом режиме ток в выходной цепи протекает в период всего действия сигнала. Режим А дает возможность получения максимальной амплитуды выходного сигнала с минимальными искажениями. Воздействие на низкоомную нагрузку RН сигнала большой амплитуды приводит к значительному увеличению КПД усилителя и его мощности.
За счет включения в коллекторную цепь предоконечного каскада, выполненного по схеме с общим эмиттером, динамической нагрузки получаем большой коэффициент усиления по напряжению. \par
Каскад охвачен местной положительной ОС, что дает возможность увеличения коэффициента усиления K, но уменьшения полосы пропускания.\par
Таким образом, основные особенности каскада предварительного усиления в том, что за счет работы в режиме А, он обеспечивает минимальные искажения, при достаточно усилении амплитуды сигнала. Включение в выходную цепь динамического сопротивления позволяет увеличить коэффициент усиления K в десятки раз.\par


\subsubsection{Ток покоя транзистора VT3:}

\begin{equation}
\label{eq:equation3_1}
 I_{\text{O К3}} = (2\ldots 3)\cdot I_{\text{Б m4}}  = (2\ldots 3)\cdot I_{\text{нм}}/h_{\text{21 экв}}
\end{equation}
\begin{equation*}
  I_{\text{O К3}} = \dfrac{2.5 \cdot 2.83 }{ 400 } = 0.02
\end{equation*}
 
 \subsubsection{Выбор резистора R7:} 
 
\begin{equation}
\label{eq:equation3_2}
R_{7} = (30\ldots 50)\cdot R_{н} = 40 \cdot 2 = 80~\text{Ом}
\end{equation}
Резистор R7 включается в цепь для того, чтобы не закорачивать источник питания конденсатором С3, обеспечивающим включение в выходную цепь транзистора динамической нагрузки R6. Основное усиление напряжения происходит за счет динамической нагрузки R6, потому резистор R7 выбирается малой величины.

\subsubsection{Выбор резистора R6:} 
\begin{equation}
\label{eq:equation3_3}
 R_{6} = (E_{0} - U_{\text{БЭ5}} - I_{\text{О К3}} \cdot R_7)/I_{\text{0 К3}} = (17 - 0.6 - 0.023 \cdot 80)/0.02 = 728~\text{Ом}
\end{equation}

При прохождении сигнала динамическое сопротивление R6 будет определяться:
\begin{equation}
\label{eq:equation3_4}
  R_{\text{6Д}} = \dfrac{R_6}{1-K_{\text{ОК}}} = 10 \cdot 728 = 7280~\text{Ом}
\end{equation}

Коэффициент усиления оконечного каскада $K_{\text{ОК}}$, т.к. он является повторителем напряжения, близок к 1 и составляет более 0.9: $K_{\text{ОК}}$ = 0.9. \par
С обеих сторон резистора R6 потенциалы близки за счет того, что цепь термостабилизации не вносит особо падения напряжения и транзисторы VT4 и VT5 являются повторителями напряжения. Ввиду этого на обоих концах установятся близкие потенциалы, т.е. разность потенциалов будет очень мала и ток практически не будет протекать. Что эквивалентно включению большого сопротивления. За счет этого происходит увеличения коэффициента усиления. \par

\subsubsection{Определение емкости C3:} 
Эта емкость устраняет протекание переменного тока по цепи R6 – R7 – земля и увеличивает коэффициент усиления каскада. Обеспечивает связь транзистора VT3 с нагрузкой R6 через оконечный каскад. \par
\begin{equation}
\label{eq:equation3_5}
  C_3 \geq \dfrac{5 \ldots 10}{2 \pi f_{\text{н}} (R_7 + R_{\text{н}})} \geq \dfrac{5}{2 \pi 10 (80 + 2)} \geq 970~\text{мкФ}
\end{equation}

\subsubsection{Параметры выбора транзистора VT3:}
\begin{equation}
\label{eq:equation3_6}
P_{\text{к доп}} = (1.2 \ldots 1.5) P_{\text{к 3}} = (1.2 \ldots 1.5) \cdot \dfrac{E_0 I_{\text{0 кз}}}{2} = 1.3 \cdot \dfrac{17 0.02}{2} = 0.221~\text{Вт}
\end{equation}

\begin{equation}
\label{eq:equation3_7}
I_{\text{к 3m}} = I_{\text{О КЗ}} + \dfrac{I_{\text{нм}}}{h_21} = 0.013 + \dfrac{2.83}{400} = 0.020~\text{А}
\end{equation}

\begin{equation}
\label{eq:equation3_8}
I_{\text{к доп}} = (1.2 \ldots 1.5) I_{\text{к 3m}} = 1.3 \cdot 0.020 = 0.026~\text{А}
\end{equation}

\begin{equation}
\label{eq:equation3_9}
U_{\text{кэ доп}} = (1.2 \ldots 1.5) E_{\text{0}} = 1.3 \cdot 17  = 22.100~\text{В}
\end{equation}

\begin{equation}
\label{eq:equation3_10}
f_{h_{21}} = (2 \ldots 3) f_{\text{в}} = 2.5 \cdot 18000 = 45~\text{кГц}
\end{equation}

Параметр h21 выбирается из максимально возможных по заданным параметрам.

\begin{table}[htbp]
\caption{Характеристики выбранного транзистора в ПОК}
\begin{center}\begin{tabular}{|c|c|c|c|c|c|c|}
\hline 
  & тип & $P_{\text{к}}$ доп, Вт & $I_{\text{к}}$ доп, А & $U_{\text{к}}$ доп, В & $h_{21}$ &  $f_{h_{21}}$, кГц \\ 
\hline 
VT4 & n-p-n & 3.66  & 3.11 & 19 & 400 & 36.00\\ 
\hline 
VT5 & p-n-p & 3.66  & 3.11 & 19 & 400 & 36.00 \\ 
\hline 
\end{tabular} 
\end{center}
\end{table}

\subsubsection{Расчет цепи смещения:}

Схема цепи смещения на транзисторах представлена на рисунке 2.4. \par
Находим ток делителя:
\begin{equation}
\label{eq:equation3_11}
 I_{\text{д}} = (0.1 \ldots 0.3) I_{\text{0кз}} = 0.2 \cdot 0.02 = 0.04~\text{А}
\end{equation}

Выбор VTt практически определяется допустимым током
\begin{equation}
\label{eq:equation3_12}
 I_{\text{к доп}} = (1.1 \ldots 1.3) I_{\text{к3max}} = 1.2 \cdot 0.024 = 0.029~\text{А}
\end{equation}

Определяем $R_{\text{бт}}$ ($U_{\text{бт}} \approx 0,5 - 0,6$ В)
\begin{equation}
\label{eq:equation3_13}
 R_{\text{бт}} = U_{\text{бт}} / I_{\text{д}} = 0.5/0.029 = 17~\text{Ом}
\end{equation}

Сопротивление подстроечного резистора
\begin{equation}
\label{eq:equation3_14}
 R_{\text{П}} = 2 (U_{\text{СМ}} - n U_{\text{Д}}) / I_{\text{0 КЗ}} = 2 (1.2 - 0.5) / 0.04 = 35~\text{Ом}
\end{equation}

\subsubsection{Входное сопротивление предоконечного каскада:}
Рассчитаем $R_{\text{ВХ3}}$ и $r_{\text{Э3}}$
\begin{equation}
\label{eq:equation3_15}
 R_{\text{ВХ3}} = h_{\text{11 VT3}} = (1 + h_{\text{21 VT3}}) \psi_T / I_{\text{0 Э3}} = (1 + 80) \cdot 25 / 40 = 51~\text{Ом}
\end{equation}
\begin{equation}
\label{eq:equation3_16}
 r_{\text{Э3}} = \psi_T / I_{\text{0 Э3}} = 25/40 = 0.625~\text{Ом}
\end{equation}

Что соответствует значению входного сопротивления в схеме с общим эмиттером, которое имеет небольшое значение и определяется сопротивлением прямо смещенного эмиттерного перехода, имеющем незначительную величину, в пересчете на малый входной ток базы.

\subsubsection{Коэффициент усиления каскада по напряжению:}

Предоконечный каскад имеет большое усиление по напряжению за счет того, что в коллекторной цепи включена динамическая нагрузка.
\begin{equation}
\label{eq:equation3_17}
 K = \dfrac{R_{\text{КН3}}}{r_{\text{э3}}} = \dfrac{R_{\text{ВХ4}} ( R_{\text{6Д}} + К_7)}{ r_{\text{э3}}} = \dfrac{h_{21} R_{н}}{r_{\text{э3}}}
\end{equation}
\begin{equation*}
  K = (400 \cdot 2) / 0.625 = 1280
\end{equation*}

Входное сопротивление оконечных каскадов состоит из малого входного сопротивления транзистора по схеме с общим эмиттером и сопротивлением, учитывающим влияние местной ООС, тем самым увеличивая входное сопротивление.  Влияние R9 и R10 , ввиду их малости, можно не учитывать. \par

\subsubsection{Итоговые данные предоконечного каскада:}

\begin{table}[htbp]
\caption{Параметры выбора транзистора}
\begin{center}\begin{tabular}{|c|c|c|c|c|c|c|}
\hline 
  & тип & $P_{\text{к}}$ доп, Вт & $I_{\text{к}}$ доп, А & $U_{\text{к}}$ доп, В & $h_{21}$ &  $f_{h_{21}}$, кГц \\ 
\hline 
VT4 & n-p-n & 3.66  & 3.11 & 19 & 400 & 36.00\\ 
\hline 
VT5 & p-n-p & 3.66  & 3.11 & 19 & 400 & 36.00 \\ 
\hline 
\end{tabular} 
\end{center}
\end{table}

\begin{table}[htbp]
\caption{Режимы работы транзистора}
\begin{center}\begin{tabular}{|c|c|c|c|c|c|c|}
\hline 
  & тип & $P_{\text{к}}$ доп, Вт & $I_{\text{к}}$ доп, А & $U_{\text{к}}$ доп, В & $h_{21}$ &  $f_{h_{21}}$, кГц \\ 
\hline 
VT4 & n-p-n & 3.66  & 3.11 & 19 & 400 & 36.00\\ 
\hline 
VT5 & p-n-p & 3.66  & 3.11 & 19 & 400 & 36.00 \\ 
\hline 
\end{tabular} 
\end{center}
\end{table}